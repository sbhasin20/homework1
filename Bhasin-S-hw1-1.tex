% Options for packages loaded elsewhere
\PassOptionsToPackage{unicode}{hyperref}
\PassOptionsToPackage{hyphens}{url}
%
\documentclass[
]{article}
\usepackage{amsmath,amssymb}
\usepackage{lmodern}
\usepackage{iftex}
\ifPDFTeX
  \usepackage[T1]{fontenc}
  \usepackage[utf8]{inputenc}
  \usepackage{textcomp} % provide euro and other symbols
\else % if luatex or xetex
  \usepackage{unicode-math}
  \defaultfontfeatures{Scale=MatchLowercase}
  \defaultfontfeatures[\rmfamily]{Ligatures=TeX,Scale=1}
\fi
% Use upquote if available, for straight quotes in verbatim environments
\IfFileExists{upquote.sty}{\usepackage{upquote}}{}
\IfFileExists{microtype.sty}{% use microtype if available
  \usepackage[]{microtype}
  \UseMicrotypeSet[protrusion]{basicmath} % disable protrusion for tt fonts
}{}
\makeatletter
\@ifundefined{KOMAClassName}{% if non-KOMA class
  \IfFileExists{parskip.sty}{%
    \usepackage{parskip}
  }{% else
    \setlength{\parindent}{0pt}
    \setlength{\parskip}{6pt plus 2pt minus 1pt}}
}{% if KOMA class
  \KOMAoptions{parskip=half}}
\makeatother
\usepackage{xcolor}
\usepackage[margin=1in]{geometry}
\usepackage{color}
\usepackage{fancyvrb}
\newcommand{\VerbBar}{|}
\newcommand{\VERB}{\Verb[commandchars=\\\{\}]}
\DefineVerbatimEnvironment{Highlighting}{Verbatim}{commandchars=\\\{\}}
% Add ',fontsize=\small' for more characters per line
\usepackage{framed}
\definecolor{shadecolor}{RGB}{248,248,248}
\newenvironment{Shaded}{\begin{snugshade}}{\end{snugshade}}
\newcommand{\AlertTok}[1]{\textcolor[rgb]{0.94,0.16,0.16}{#1}}
\newcommand{\AnnotationTok}[1]{\textcolor[rgb]{0.56,0.35,0.01}{\textbf{\textit{#1}}}}
\newcommand{\AttributeTok}[1]{\textcolor[rgb]{0.77,0.63,0.00}{#1}}
\newcommand{\BaseNTok}[1]{\textcolor[rgb]{0.00,0.00,0.81}{#1}}
\newcommand{\BuiltInTok}[1]{#1}
\newcommand{\CharTok}[1]{\textcolor[rgb]{0.31,0.60,0.02}{#1}}
\newcommand{\CommentTok}[1]{\textcolor[rgb]{0.56,0.35,0.01}{\textit{#1}}}
\newcommand{\CommentVarTok}[1]{\textcolor[rgb]{0.56,0.35,0.01}{\textbf{\textit{#1}}}}
\newcommand{\ConstantTok}[1]{\textcolor[rgb]{0.00,0.00,0.00}{#1}}
\newcommand{\ControlFlowTok}[1]{\textcolor[rgb]{0.13,0.29,0.53}{\textbf{#1}}}
\newcommand{\DataTypeTok}[1]{\textcolor[rgb]{0.13,0.29,0.53}{#1}}
\newcommand{\DecValTok}[1]{\textcolor[rgb]{0.00,0.00,0.81}{#1}}
\newcommand{\DocumentationTok}[1]{\textcolor[rgb]{0.56,0.35,0.01}{\textbf{\textit{#1}}}}
\newcommand{\ErrorTok}[1]{\textcolor[rgb]{0.64,0.00,0.00}{\textbf{#1}}}
\newcommand{\ExtensionTok}[1]{#1}
\newcommand{\FloatTok}[1]{\textcolor[rgb]{0.00,0.00,0.81}{#1}}
\newcommand{\FunctionTok}[1]{\textcolor[rgb]{0.00,0.00,0.00}{#1}}
\newcommand{\ImportTok}[1]{#1}
\newcommand{\InformationTok}[1]{\textcolor[rgb]{0.56,0.35,0.01}{\textbf{\textit{#1}}}}
\newcommand{\KeywordTok}[1]{\textcolor[rgb]{0.13,0.29,0.53}{\textbf{#1}}}
\newcommand{\NormalTok}[1]{#1}
\newcommand{\OperatorTok}[1]{\textcolor[rgb]{0.81,0.36,0.00}{\textbf{#1}}}
\newcommand{\OtherTok}[1]{\textcolor[rgb]{0.56,0.35,0.01}{#1}}
\newcommand{\PreprocessorTok}[1]{\textcolor[rgb]{0.56,0.35,0.01}{\textit{#1}}}
\newcommand{\RegionMarkerTok}[1]{#1}
\newcommand{\SpecialCharTok}[1]{\textcolor[rgb]{0.00,0.00,0.00}{#1}}
\newcommand{\SpecialStringTok}[1]{\textcolor[rgb]{0.31,0.60,0.02}{#1}}
\newcommand{\StringTok}[1]{\textcolor[rgb]{0.31,0.60,0.02}{#1}}
\newcommand{\VariableTok}[1]{\textcolor[rgb]{0.00,0.00,0.00}{#1}}
\newcommand{\VerbatimStringTok}[1]{\textcolor[rgb]{0.31,0.60,0.02}{#1}}
\newcommand{\WarningTok}[1]{\textcolor[rgb]{0.56,0.35,0.01}{\textbf{\textit{#1}}}}
\usepackage{graphicx}
\makeatletter
\def\maxwidth{\ifdim\Gin@nat@width>\linewidth\linewidth\else\Gin@nat@width\fi}
\def\maxheight{\ifdim\Gin@nat@height>\textheight\textheight\else\Gin@nat@height\fi}
\makeatother
% Scale images if necessary, so that they will not overflow the page
% margins by default, and it is still possible to overwrite the defaults
% using explicit options in \includegraphics[width, height, ...]{}
\setkeys{Gin}{width=\maxwidth,height=\maxheight,keepaspectratio}
% Set default figure placement to htbp
\makeatletter
\def\fps@figure{htbp}
\makeatother
\setlength{\emergencystretch}{3em} % prevent overfull lines
\providecommand{\tightlist}{%
  \setlength{\itemsep}{0pt}\setlength{\parskip}{0pt}}
\setcounter{secnumdepth}{-\maxdimen} % remove section numbering
\ifLuaTeX
  \usepackage{selnolig}  % disable illegal ligatures
\fi
\IfFileExists{bookmark.sty}{\usepackage{bookmark}}{\usepackage{hyperref}}
\IfFileExists{xurl.sty}{\usepackage{xurl}}{} % add URL line breaks if available
\urlstyle{same} % disable monospaced font for URLs
\hypersetup{
  pdftitle={Bhasin-S-hw1-1},
  hidelinks,
  pdfcreator={LaTeX via pandoc}}

\title{Bhasin-S-hw1-1}
\author{}
\date{\vspace{-2.5em}2023-01-23}

\begin{document}
\maketitle

\hypertarget{r-markdown}{%
\subsection{R Markdown}\label{r-markdown}}

This is an R Markdown document. Markdown is a simple formatting syntax
for authoring HTML, PDF, and MS Word documents. For more details on
using R Markdown see \url{http://rmarkdown.rstudio.com}.

When you click the \textbf{Knit} button a document will be generated
that includes both content as well as the output of any embedded R code
chunks within the document. You can embed an R code chunk like this:

\begin{Shaded}
\begin{Highlighting}[]
\FunctionTok{summary}\NormalTok{(cars)}
\end{Highlighting}
\end{Shaded}

\begin{verbatim}
##      speed           dist       
##  Min.   : 4.0   Min.   :  2.00  
##  1st Qu.:12.0   1st Qu.: 26.00  
##  Median :15.0   Median : 36.00  
##  Mean   :15.4   Mean   : 42.98  
##  3rd Qu.:19.0   3rd Qu.: 56.00  
##  Max.   :25.0   Max.   :120.00
\end{verbatim}

\hypertarget{including-plots}{%
\subsection{Including Plots}\label{including-plots}}

You can also embed plots, for example:

\includegraphics{Bhasin-S-hw1-1_files/figure-latex/pressure-1.pdf}

Note that the \texttt{echo\ =\ FALSE} parameter was added to the code
chunk to prevent printing of the R code that generated the plot.

\hypertarget{read-in-enrollment-data-for-january-of-each-year}{%
\subsection{Read in enrollment data for january of each
year}\label{read-in-enrollment-data-for-january-of-each-year}}

install.packages(``usethis'') install.packages(``tidyverse'')
library(tidyverse)

for (y in 2007:2015) \{ \#\# Basic contract/plan information
ma.path=paste0(``data/input/monthly-ma-and-pdp-enrollment-by-cpsc/CPSC\_Contract\_Info\_'',y,``\_01.csv'')
contract.info=read\_csv(ma.path, skip=1, col\_names =
c(``contractid'',``planid'',``org\_type'',``plan\_type'',
``partd'',``snp'',``eghp'',``org\_name'',``org\_marketing\_name'',
``plan\_name'',``parent\_org'',``contract\_date''), col\_types = cols(
contractid = col\_character(), planid = col\_double(), org\_type =
col\_character(), plan\_type = col\_character(), partd =
col\_character(), snp = col\_character(), eghp = col\_character(),
org\_name = col\_character(), org\_marketing\_name = col\_character(),
plan\_name = col\_character(), parent\_org = col\_character(),
contract\_date = col\_character() ))

contract.info = contract.info \%\textgreater\% group\_by(contractid,
planid) \%\textgreater\% mutate(id\_count=row\_number())

contract.info = contract.info \%\textgreater\% filter(id\_count==1)
\%\textgreater\% select(-id\_count)

\#\# Enrollments per plan
ma.path=paste0(``data/input/monthly-ma-and-pdp-enrollment-by-cpsc/CPSC\_Enrollment\_Info\_'',y,``\_01.csv'')
enroll.info=read\_csv(ma.path, skip=1, col\_names =
c(``contractid'',``planid'',``ssa'',``fips'',``state'',``county'',``enrollment''),
col\_types = cols( contractid = col\_character(), planid =
col\_double(), ssa = col\_double(), fips = col\_double(), state =
col\_character(), county = col\_character(), enrollment = col\_double()
),na=``*``)

\#\# Merge contract info with enrollment info plan.data = contract.info
\%\textgreater\% left\_join(enroll.info, by=c(``contractid'',
``planid'')) \%\textgreater\% mutate(year=y)

\#\# Fill in missing fips codes (by state and county) plan.data =
plan.data \%\textgreater\% group\_by(state, county) \%\textgreater\%
fill(fips)

\#\# Fill in missing plan characteristics by contract and plan id
plan.data = plan.data \%\textgreater\% group\_by(contractid, planid)
\%\textgreater\% fill(plan\_type, partd, snp, eghp, plan\_name)

\#\# Fill in missing contract characteristics by contractid plan.data =
plan.data \%\textgreater\% group\_by(contractid) \%\textgreater\%
fill(org\_type,org\_name,org\_marketing\_name,parent\_org)

\#\# Collapse from monthly data to yearly plan.year = plan.data
\%\textgreater\% group\_by(contractid, planid, fips) \%\textgreater\%
arrange(contractid, planid, fips) \%\textgreater\%
rename(avg\_enrollment=enrollment)

write\_rds(plan.year,paste0(``data/output/ma\_data\_'',y,``.rds'')) \}

full.ma.data \textless- read\_rds(``data/output/ma\_data\_2007.rds'')
for (y in 2008:2015) \{ full.ma.data \textless-
rbind(full.ma.data,read\_rds(paste0(``data/output/ma\_data\_'',y,``.rds'')))
\}

write\_rds(full.ma.data,``data/output/full\_ma\_data.rds'')
sapply(paste0(``ma\_data\_'', 2007:2015, ``.rds''), unlink)

\#Homework 1

\#Enrollment Data

\#1. There are 19,126,783 observations in my current data set.

full.ma.data \%\textgreater\% count(plan\_type)

\#2 There are 5,847,057 different plan\_type in the data

file.path(plan.data)

plan.data \textless- read.csv(``plan.data.csv'') plan\_type \textless-
plan.data\$plan\_type

knitr::kable(plan.type,
col.names=c(``2010'',``2011'',``2012'',``2013'',``2014'',``2015''),
type=``html'', caption = ``Plan Count by Year'', booktabs = TRUE)

\#3

full.ma.data \textless- readRDS(`data/full.ma.data.rds')

\end{document}
